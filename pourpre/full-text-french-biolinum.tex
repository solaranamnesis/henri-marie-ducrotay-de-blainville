\documentclass[a4paper, 11pt, oneside, polutonikogreek, french]{article}
\usepackage[sfdefault]{biolinum}
\usepackage[LGR,T1]{fontenc}

% Load encoding definitions (after font package)

\usepackage{textalpha}
\usepackage{bbding}

\usepackage{listings}
\lstset{basicstyle=\ttfamily}

% Babel package:
\usepackage[french]{babel}

% With XeTeX$\$LuaTeX, load fontspec after babel to use Unicode
% fonts for Latin script and LGR for Greek:
\ifdefined\luatexversion \usepackage{fontspec}\fi
\ifdefined\XeTeXrevision \usepackage{fontspec}\fi

% "Lipsiakos" italic font `cbleipzig`:
\newcommand*{\lishape}{\fontencoding{LGR}\fontfamily{cmr}%
		       \fontshape{li}\selectfont}
\DeclareTextFontCommand{\textli}{\lishape}

\usepackage{booktabs}
\usepackage{graphicx}
\setlength{\emergencystretch}{15pt}
\graphicspath{ {./ } }
\usepackage[figurename=]{caption}
\usepackage{float}
\usepackage{fancyhdr}
\usepackage{microtype}
\begin{document}
\begin{titlepage} % Suppresses headers and footers on the title page
	\centering % Centre everything on the title page
	%\scshape % Use small caps for all text on the title page

	%------------------------------------------------
	%	Title
	%------------------------------------------------
	
	\rule{\textwidth}{1.6pt}\vspace*{-\baselineskip}\vspace*{2pt} % Thick horizontal rule
	\rule{\textwidth}{0.4pt} % Thin horizontal rule
	
	\vspace{1\baselineskip} % Whitespace above the title
	
	{\scshape\Huge Pourpre}
	
	\vspace{1\baselineskip} % Whitespace above the title

	\rule{\textwidth}{0.4pt}\vspace*{-\baselineskip}\vspace{3.2pt} % Thin horizontal rule
	\rule{\textwidth}{1.6pt} % Thick horizontal rule
	
	\vspace{1\baselineskip} % Whitespace after the title block
	
	%------------------------------------------------
	%	Subtitle
	%------------------------------------------------
	
	{\scshape Par \Large Henri-Marie Ducrotay de Blainville} % Subtitle or further description
	
	\vspace*{1\baselineskip} % Whitespace under the subtitle
    
	%------------------------------------------------
	%	Editor(s)
	%------------------------------------------------
        \vspace*{\fill}

	\vspace{1\baselineskip}

	{\small\scshape Paris 1826 \\ Extrait du 43\textsuperscript{e} volume de \emph{Dictionnaire des Sciences Naturelles}}
		
	\vspace{0.5\baselineskip} % Whitespace after the title block

        \scshape Internet Archive Online Edition  % Publication year
	
	{\scshape\small Utilisation non commerciale --- Partage dans les mêmes conditions 4.0 International} % Publisher
\end{titlepage}
\setlength{\parskip}{1mm plus1mm minus1mm}
\clearpage
\paragraph{}
\textbf{Pourpre}, \emph{Purpura}. (\emph{Conchyl.}) M. de Lamarck est le premier zoologiste qui ait établi sous ce nom un genre distinct, quoique peut-être encore mal circonscrit, pour un assez grand nombre d'espèces de coquilles, que Linné et ceux qui ont suivi son système, plaçaient dans les genres Buccin et Rocher (\emph{Murex}). Cette dénomination de pourpre se trouve cependant chez un assez grand nombre de conchyliologistes antérieurs à M. de Lamarck ; mais ils l'appliquaient d'une manière toute différente. Adanson, par exemple, désigne ainsi le second genre de sa section des limaçons operculés ; mais il y comprend non-seulement les pourpres proprement dites, mais encore les rochers, les buccins, les strombes de Linné, et de plus les tonnes, les casques, les cassidaires, les cancellaires, et même les volutes des conchyliologistes modernes. Il est bien vrai qu'il les divise en plusieurs sections fort naturelles, d'après la forme de l'ouverture, mais surtout d'après la longueur du canal qui la termine ; cependant il est évident qu'il en résultait une assez grande confusion. Bruguière, qui, le premier, a entrepris la réforme de la conchyliologie, confondait les pourpres parmi ses buccins. Quant aux anciens conchyliologistes, comme Rondelet, Rumph, Séba, Gualtieri, d'Argenville, Favannes, etc., on voit qu'en général ils appliquaient la dénomination de pourpre plutôt à des murex qu'à des buccins, ce qui est à peu près le contraire aujourd'hui. Cette grande extension, que les auteurs ont ainsi donnée au nom de pourpre, tient sans doute à la juste observation, que tous ces animaux fournissent en plus ou moins grande abondance la matière que les anciens employaient pour teindre les étoffes en pourpre ; aussi chacun d'eux a-t-il donné une définition assez différente de la division des pourpres. Voici celle que nous avons établie, et qui est tirée de la considération de l'animal, de la coquille et même de son opercule : Animal peu alongé, élargi en avant ; manteau à bords simples et pourvu d'un tube distinct ; pied fort large, elliptique, très-avancé, subbilobé en avant, et portant à la face dorsale de sa partie postérieure un assez grand opercule ; tête large, peu distincte ; tentacules antérieurs très-rapprochés à la base, subcylindriques, et portant les yeux aux deux tiers de leur longueur, beaucoup plus renflés que le reste ; bouche inférieure, cachée par la grande avance du pied ; deux branchies pectiniformes, presque parallèles ; la droite plus grande que la gauche ; l'appareil générateur comme dans les buccins. Coquille épaisse, ovale, à spire courte ; le dernier tour beaucoup plus grand que tous les autres réunis ; ouverture ovale, très-évasée, terminée en avant par un canal fort court, oblique et échancré à son extrémité ; le bord columellaire, presque droit, et chargé d'une sorte de callosité pointue en avant : opercule corné, plat, presque semi-circulaire, à stries peu marquées et transverses ; le sommet en arrière. Comme nous avons caractérisé ce genre essentiellement d'après un individu de la pourpre persique, rapporté par MM. Quoy et Gaimard, nous devons ajouter que dans la rigueur conchyliologique de ce genre, tel qu'il est établi par M. de Lamarck, il faut surtout avoir égard à la forme de la columelle, qui est toujours plus droite, plus large, plus aplatie, et surtout plus atténuée ou pointue en avant que dans les buccins. Quant à l'indice du canal qui termine l'ouverture, et que M. de Lamarck regarde comme conduisant aux harpes, aux tonnes et aux buccins, où il admet une disparution complète de cette partie, il faut convenir que ce caractère est encore moins tranché. En effet, pour prendre un exemple bien connu, certainement il n'y a pas plus ni moins de canal dans le \emph{buccinum lapillus}, espèce de pourpre pour M. de Lamarck, que dans le \emph{buccinum undatum}. Malgré cette ressemblance assez évidente dans la forme de la columelle du genre Pourpre, il faut convenir qu'il réunit des espèces qui paraissent assez hétérogènes, les unes étant presque lisses, tandis que d'autres sont seulement cerclées par de fortes bandes décurrentes, ou même sont hérissées de séries de tubercules pointus, absolument comme dans les ricinules. La forme particulière de l'opercule, telle que je l'ai observée dans la pourpre persique, indique bien qu'il y a réellement un genre à former avec quelques-unes des espèces que M. de Lamarck réunit dans son genre Pourpre ; mais comme il n'est guère possible de le supposer de cette forme que dans le \emph{P. patula}, et peut-être dans une ou deux autres, on doit convenir que cette réunion est véritablement artificielle, même en pure ou simple conchyliologie. En effet, il ne faudrait pas argumenter que c'est dans les espèces de ce genre que l'on trouve principalement la pourpre ; car on la trouve dans les murex comme dans les buccins, et il est même probable, comme nous allons le voir tout à l'heure, que l'espèce de coquillage dont les anciens extrayaient la couleur pourpre, n'appartenait pas au genre que M. de Lamarck a exclusivement désigné par ce nom.

L'organisation des pourpres ne diffère en effet presque en aucune manière de celle des buccins et des rochers, si ce n'est peut-être dans la grandeur du pied et dans la forme de l'opercule, puisqu'ils ont également un manteau à bord simple, avec un siphon respiratoire ; que la tête, pourvue d'une paire de tentacules grêles, sétacés, porte les yeux dans une partie de leur étendue ; que la bouche est armée d'une longue trompe avec des crochets ; que les branchies, toujours pectiniformes, sont également au nombre de deux ; que les sexes sont séparés sur des individus différents, et que l'organe excitateur mâle est aussi constamment extérieur et caché sous le manteau. Quant à l'organe qui produit la pourpre, c'est un petit sac que l'on trouve dans tous les siphonobranches, et qui se dirige obliquement de gauche à droite ; mais ce petit sac, ou mieux ce canal, n'est que le canal excréteur de l'organe que M. Cuvier a nommé l'organe de la viscosité, et qui, placé vers le cœur, entre lui et le rectum, paraît être une sorte d'appareil dépurateur ou urinaire.

Les mœurs et les habitudes des pourpres sont, sans aucune espèce de doute, semblables à celles des buccins. Ce sont des animaux marins, qui vivent dans les anfractuosités des rochers, dans les lieux couverts de fucus, mais qui peuvent aussi s'enfoncer dans le sable. Ils rampent, à l'aide de leur pied, comme les autres gastéropodes. Leur nourriture paraît être constamment animale et être obtenue en perçant la coquille, principalement celle des animaux mollusques bivalves. Nous savons, par exemple, que la pourpre des teinturiers, \emph{P. lapillus}, se nourrit essentiellement sur nos côtes de la chair du \emph{lepas balanoides}, L., \emph{balanus striatus} des zoologistes modernes.

Le mode d'accouplement n'est pas connu. L'on ignore de même si les individus femelles déposent leurs œufs un à un, comme le fait le \emph{P. lapillus} de nos côtes. Ces œufs sphéroïdes, un peu alongés, cornés, de couleur jaunâtre, ne sont déposés que vers l'automne. Ils sont adhérents aux rochers et à d'autres corps submergés, par une espèce d'empâtement inférieur. Leur autre extrémité est fermée par une sorte de petit opercule ovale, épais, transverse. Quant à leur intérieur , on y trouve une matière plus épaisse au milieu d'une autre plus fluide. Malheureusement on n'en a pas suivi suffisamment les développements, et même, pendant assez longtemps, ces œufs vides ont été regardés comme une espèce d'hydre, \emph{H. triticea}.

Les espèces de pourpre, en admettant ce genre tel que M. de Lamarck l'a circonscrit, se trouvent dans toutes les mers ; mais le plus grand nombre et les plus grosses espèces proviennent toujours des mers des pays chauds, et surtout des mers australes. On peut les partager en plusieurs sections assez tranchées, en ayant égard à l'aspect général de la coquille. Mais, avant de passer à l'indication des espèces, nous allons faire une digression sur la pourpre des anciens.

Le mot de pourpre, πορφύρα en grec, \emph{purpura} en latin, est employé indifféremment par les anciens pour désigner, 1.° la couleur rouge-bleue que nous connaissons aujourd'hui sous le même nom, quoique nous l'obtenions d'une toute autre manière ; 2.° l'animal qui la leur fournissait. Il est cependant probable que c'est de la couleur qu'il a passé à celui-ci. Je ne m'amuserai pas à en rechercher l'étymologie ; mais je vais voir, s'il est possible, de reconnaître l'espèce de mollusques que les anciens employaient pour en obtenir la pourpre, les procédés qu'ils suivaient dans cet emploi ; après quoi je chercherai si nous pourrions donner aujourd'hui la théorie de la purpurification.

Aristote est le premier auteur qui ait parlé de la pourpre. Voici ce qu'il en dit :

« La pourpre, si ce n'est la tête, a toutes ses autres parties contenues dans la coquille ; elle est pourvue d'une trompe très-ferme, qui lui tient lieu de langue et dont elle se sert pour percer le test de l'animal, dont elle fait sa nourriture. Dans la partie turbinée de la coquille sont contenus l'estomac, le foie et l'intestin ; l'extrémité de ce dernier aboutit auprès de la tête. C'est entre le cou et le foie que se trouve l'organe qui fournit, quand on l'écrase, la matière colorante ; il a la forme d'une veine : ce qui remplit le reste de l'intervalle ressemble à de l'alun. » Il est probable qu'ici Aristote fait allusion à la matière crétacée qu'on trouve fréquemment gorgeant le rectum de beaucoup de mollusques.

« Les pourpres se meuvent peu ; elles se tiennent cachées pendant les grandes chaleurs de la canicule. L'eau douce leur est absolument nuisible ; mais elles peuvent vivre jusqu'à cinquante jours hors de l'eau : elles sentent leur proie de très loin. Elles se rassemblent au printemps dans le même endroit et y font ce qu'on nomme leur \emph{cire}, c'est-à-dire une production qui ressemble à un gâteau de cire, si ce n'est qu'elle n'est pas lisse, ou mieux à une multitude de cosses de pois blancs réunies. On n'y aperçoit jamais d'ouverture. Quand les pourpres, comme les autres testacés, commencent à former cette production, ils produisent une mucosité gluante qui sert à lier ces espèces de cosses. C'est dans cette masse réunie que naissent les petites pourpres, que l'on trouve attachées, quelquefois encore informes, à la coquille des grandes qu'on pêche. Si l'on prend les pourpres avant qu'elles n'aient jeté ce \emph{frai}, elles le font dans les paniers où elles se trouvent, et l'espace étroit dans lequel elles sont renfermées, donne seulement à la masse de la \emph{cire} la forme d'une grappe de raisin. »

Quoique Aristote ait prétendu que dans ces animaux il n'y ait pas de génération proprement dite, mais qu'ils naissent de la terre, il est évident que cette description convient parfaitement bien aux œufs des pourpres, qui par conséquent ressemblent beaucoup à ceux de notre \emph{buccinum undatum}.

« On distingue plusieurs espèces de pourpres ; aussi il y en a de grandes, comme celles du promontoire de Siget et de celui de Lecte, et de petites, comme celles de l'Euripe et des côtes de Carie : en général, celles qui se pêchent dans les golfes sont grandes et d'une surface inégale ; il y en a qui pèsent jusqu'à une mine. » La couleur qu'elles fournissent, et qu'Aristote nomme \emph{fleur} (ἀνθος), est le plus souvent noire, quelquefois rouge et en petite quantité. Sur les rivages et autour des promontoires elles sont petites et leur liqueur est rouge. Dans les lieux exposés au nord, elle est en général noire, et rouge dans ceux qui le sont au midi. Elle n'est jamais moins bonne que lorsque les pourpres ont jeté leur frai ; aussi les pêche-t-on au printemps dans le moment qu'elles s'en débarrassent. On les prenait autrefois au moyen d'appâts, composés de chair qui se gâte ou de petits poissons, et sans filet ; mais, comme souvent elles retombaient dans l'eau après en avoir été tirées, pour éviter cet inconvénient, les pêcheurs mettent des nasses au-dessous et autour de l'appât, de manière que si elles viennent à tomber, ce qu'elles font aisément quand elles en sont rassasiées (car avant il est difficile de les arracher), elles ne sont pas perdues : on les laisse ensuite dans les nasses, où on les prend, jusqu'à ce qu'on en ait une quantité suffisante et qu'on puisse les employer. Pour en extraire la liqueur, on enlève, du moins pour les grosses, l'animal de sa coquille et ensuite on prend la partie située entre le cou et le foie ou la veine ; mais, pour les petits individus, on les concasse avec leur coquille, parce qu'il y aurait trop de difficulté de les en séparer ; « mais, ajoute Aristote, on a soin de le faire quand elles sont vivantes, sans quoi, si elles mouraient naturellement, elles jetteraient leur liqueur en expirant. »

Pline abrège considérablement ce que dit Aristote de la pourpre et même en le modifiant d'une manière à peu près inintelligible, ce qui prouve qu'il n'a pas compris le texte d'Aristote ; il ajoute « qu'elles vivent ordinairement sept ans, quoiqu'elles croissent encore plus promptement que les autres coquillages, et qu'elles aient atteint toute leur croissance au bout d'un an ; qu'elles peuvent vivre jusqu'à cinquante jours sans manger ; qu'elles restent cachées environ trente jours pendant la canicule, et que c'est surtout sur les rivages de Tyr en Asie, de Meniux et de Gitulor en Afrique, et de la Laconie en Europe, où l'on trouve la plus belle pourpre. » Mais il ajoute plusieurs choses intéressantes sur les espèces de coquillages dont on tirait des matières colorantes différentes.

« Il y a deux genres de coquilles qui fournissent les couleurs pourpres et les couleurs conchyliennes, couleurs qui ne diffèrent que par la nuance. La plus petite est un buccin, ainsi nommée parce qu'elle ressemble un peu à celle dont on tire le son du buccin : son ouverture est ronde et son bord est échancré ; elle ne s'attache qu'aux pierres et on la trouve autour des rochers. L'autre se nomme pourpre : elle est en forme de massue et composée de sept tours de spire, qui indiquent les années, comme dans le buccin ; mais elle est hérissée d'aiguillons, qui n'existent plus dans celui-ci ; elle est en outre pourvue d'un rostre saillant en forme de tube, et sur les côtés de petites épines tuberculeuses dans lesquelles l'animal peut introduire sa langue. »

« On distingue aussi les pourpres par la dénomination de pélagiennes, parmi lesquelles on établit plusieurs variétés, d'après les lieux qu'elles habitent et les substances dont elles se nourrissent. Celles qui vivent dans la vase ou parmi les algues et qui s'en nourrissent, sont très-peu estimées ; celles des rivages, recueillies sur les bords de la mer, sont meilleures, quoique la couleur qu'elles fournissent soit plus légère et plus claire. Une autre variété, qui est appelée graveleuse, à cause des graviers de la mer où on la trouve, est extrêmement propre pour les couleurs conchyliennes ; mais la meilleure pour les couleurs pourpres est la \emph{dialutense}, c'est-à-dire celle qui se nourrit de différentes sortes de terrain, ou mieux sur différentes sortes de terrain. »

« On prend les pourpres à l'aide de petits filets, des espèces de nasses à mailles peu serrées, que l'on jette dans la haute mer. On y met pour appât des coquillages bivalves, susceptibles de s'ouvrir et de se fermer, ou des moules, qui, à demi mortes, se raniment aussitôt qu'on les rend à la mer, et entrouvrent leur coquille. Les pourpres, avides de s'en nourrir, les attaquent en y enfonçant leur trompe ; mais bientôt, stimulées par cette aiguillon, les moules se referment et retiennent les pourpres qui les mordent, en sorte que, victimes de leur avidité, celles-ci sont enlevées encore suspendues à leur proie. L'époque la plus avantageuse pour faire cette pêche, est après le lever de la canicule, ou avant le printemps, parce que, lorsque les pourpres ont frayé, leurs sucs sont trop liquides (ou peut-être trop peu tenaces, comme le veut Gronov) ; mais c'est ce qu'ignorent les ouvriers, quoique cela soit fort essentiel. »

« Pour employer les pourpres à la teinture, on commence par leur enlever la veine dont il a été parlé plus haut, et on ajoute à cent livres de cette matière vingt onces de sel. On laisse le tout macérer pendant trois jours au juste ; car l'action est d'autant plus grande, qu'elle est plus récente. (Gronow donne à cette phrase un tout autre sens : égaler cent cinquante livres du bain à chaque ou pour chaque amphore d'eau.) On fait bouillir dans une chaudière de plomb, jusqu'à réduction de cent amphores de matière à cinq cents livres. On entretient ensuite une chaleur modérée, et cela au foyer d'un long fourneau ; après quoi les chairs, qui ont dû nécessairement rester attachées aux veines, étant écumées, et la teinture étant complétement liquéfiée au dixième jour, et ensuite soutirée, on y plonge la laine ; l'on continue à chauffer jusqu'à ce qu'on ait atteint le point désiré. Une teinte rouge vif vaut moins qu'une rouge noirâtre. On laisse ainsi la laine s'imbiber pendant cinq heures ; puis, après l'avoir cardée, on la replonge dans le bain, jusqu'à ce qu'elle ait bu autant de liqueur que possible. Le buccin ne s'emploie jamais seul, parce qu'il fournit une teinture qui ne tient pas, ou peut-être parce qu'il ne conserve pas une teinte rouge vif ; mais, en le mêlant avec la pourpre, il donne à la teinte trop noire de celle-ci cette solidité, cet éclat de l'écarlate que l'on recherche. Par ce mélange, ces couleurs s'avivent ou s'amortissent l'une l'autre. La proportion la meilleure est celle où, pour cinquante livres de laine, on emploie deux cents livres de buccin et cent onze livres de pourpre. C'est ainsi que l'on obtient cette superbe couleur que l'on nomme améthyste ; pour obtenir la couleur tyrienne, on sature d'abord la laine dans un bain encore vert et non noir de pourpre, et bientôt après on la change dans le buccin : c'est la plus belle pourpre de couleur de sang figé, noirâtre vue en face, et brillante vue de côté. C'est de là qu'Homère donne au sang l'épithète de pourpre. »

« La couleur conchylienne s'obtient par les mêmes procédés, si ce n'est qu'on ne se sert pas du buccin, et qu'en outre on adoucit le bain, en en composant la moitié de parties égales d'eau et d'urine. C'est ainsi qu'on obtient, par une saturation incomplète, cette couleur pâle tant vantée et d'autant plus étendue, que la laine est pour ainsi dire moins rassasiée. »

« On obtient encore une autre teinte, que l'on a nommée tyriaméthyste, en saturant une étoffe d'abord améthystée dans un bain de pourpre de Tyr, ce qu'indique le nom ; en sorte qu'on ne teint d'abord en conchylienne, que pour faciliter la teinture tyrienne, et alors celle-ci devient, dit-on, plus agréable et plus douce ; de même que, pour obtenir le ponceau, on reteint dans la pourpre de Tyr ce qu'on avait d'abord teint dans le kermès. »

« Le prix de ces différents bains colorants est d'autant plus vil que les rivages sont plus fertiles en animaux qui la fournissent ; cependant jamais cent livres de pourpre ne coûtent plus de cinquante sesterces (onze francs cinquante centimes de notre monnaie), et de buccin plus de cent, (vingt-trois francs de notre monnaie), comme le savent ceux qui se livrent à ces dépenses énormes. »

Malgré cela, Pline ajoute plus haut, d'après Cornélius Népos, que dans la jeunesse de celui-ci, qui mourut sous Auguste, la pourpre violette, qui était à la mode, valait cent deniers, qu'on estime à quatre-vingt-dix francs de notre monnaie ; que bientôt après on préféra la pourpre rouge de Tarente, et ensuite la double pourpre de Tyr, dont la livre coûtait plus de mille deniers, ou neuf cents francs.

Vitruve rapporte aussi quelque chose des couleurs qu'il nomme ostre et pourpre, \emph{ostrum} et \emph{purpura} ; mais il se borne à dire que la première se tire d'un coquillage marin, qui fournit la pourpre, sans en donner aucun caractère, ajoutant que la couleur n'est pas toujours la même, et qu'elle varie suivant les localités, et surtout selon le cours du soleil ; aussi, dit-il, que les animaux en fournissent de noire, quand ils proviennent du Pont et de la Gaule ; entre le Nord et l'Occident elle est livide ; elle est violette vers l'Orient et l'Occident équinoxial ; enfin, celle qui vient des pays méridionaux est rouge, comme par exemple de l'île de Rhodes et des environs. Quand on recueille ces coquillages et qu'on les coupe circulairement, il en sort une sanie pourpre, qui est la même que celle que l'on obtient en broyant ces animaux dans un mortier, et comme on la retire des têts de coquillages marins, d'où vient le nom d'\emph{ostre}, on a donné ce nom à cette couleur.

Oppien, dans le cinquième livre de son Haliéticon, donne le procédé pour prendre les pourpres, basé sur leur gloutonnerie, et qui consiste à mettre pour appât, dans de petites nasses d'un tissu serré et formées d'osier, des strombes ou des cames. La pourpre alors ayant passé sa trompe très-atténuée à travers les mailles de la nasse, ne peut plus l'en retirer à cause de son gonflement, et elle reste prise.

Élien nous a laissé aussi quelques détails sur les pourpres, mais aucun qui puisse servir à décider la question de l'espèce. Dans l'un de ses chapitres il parle de la manière de les prendre à l'aide d'un filet à mailles serrées, dans lequel on place pour appât un strombe, duquel la pourpre, ayant étendu considérablement sa trompe, ne peut la retirer ; et dans un autre il dit « que, si l'on veut se servir des pourpres pour la teinture et non pour manger, il faut avoir soin de tuer l'animal d'un seul coup de pierre, afin que, mourant sans agonie, la matière colorante n'ait pas le temps d'être absorbée et perdue dans toute la masse du corps. »

Pollux nous rapporte une méthode de prendre les pourpres, qui devait être beaucoup plus productive que les autres, et qui était employée par les Phéniciens : ils se servaient d'une grande corde forte, épaisse, garnie dans toute sa longueur, à d'assez petites distances, de petites nasses, ou paniers faits d'osier ou de jonc, dont l'ouverture était rétrécie par des brins libres et convergeant dans l'intérieur, de manière à se toucher. Chacune de ces masses étant amorcée, les purpuriens allaient jeter leur corde dans des lieux rocailleux, en ayant soin de mettre un morceau de liége à une extrémité, pour la reconnaître ; après une ou plusieurs nuits d'exposition, ils retiraient leurs nasses remplies de pourpres.

Ainsi, d'après ce que nous venons de voir, aucun auteur ancien n'a donné, sur les coquillages dont ils tiraient la pourpre, des détails suffisants pour qu'il soit possible de déterminer au juste quelle est l'espèce actuellement existante qu'ils ont employée. Les modernes ont-ils été plus heureux ?

Belon donne une description extérieure et même intérieure assez exacte d'un mollusque qu'il regarde comme la pourpre d'Aristote, et qui, en effet, ne doit pas en différer beaucoup ; mais la coquille qu'il figure comme la petite espèce, paraît être un jeune strombe des conchyliologistes modernes, et celle qu'il donne pour la grande espèce est évidemment un ptérocère.

Rondelet, quoiqu'il ait critiqué à tort Belon sur ce qu'il dit de l'organisation des pourpres, a peut-être été plus heureux que lui, en regardant comme la pourpre des anciens le \emph{murex brandaris} de Linné, qui paraît assez commun dans la Méditerranée, et dont la coquille offre assez bien les caractères de la pourpre d'Aristote. Son cor de mer ou son buccin, dont il dit que la couleur est moins vive et moins estimée que celle des pourpres, n'est qu'une espèce de triton de M. de Lamarck, \emph{murex lampas}, L., et son \emph{conchylium} me semble n'être qu'un jeune strombe.

Gesner et Aldrovande n'ont fait que compiler avec beaucoup de soin tout ce qui avait été dit avant eux sur la pourpre et la couleur qu'elle fournit, mais sans éclaircir le moins du monde la matière.

Fabius Columna, dans un travail \emph{ex professo} sur ce sujet (\emph{De purpurâ}, \emph{Roma} 1616), établit d'abord que les mots \emph{conchylium} et \emph{murex} sont synonymes de \emph{purpura}, et que ce mot est aussi bien pris pour la couleur que pour l'animal qui la fournit ; ensuite il décrit et figure comme tel une espèce de rocher, \emph{murex trunculus}, L., commune dans la Méditerranée et qui fournit en effet une très-grande quantité de matière colorante : elle se trouve surtout très-abondamment dans les lieux rocailleux, auprès de Naples, où les pêcheurs la prennent avec des morceaux de poumons placés sur des claies de jonc. On en trouve également beaucoup auprès du cap Misène, dans la mer actuellement appelée \emph{Mare mortuum} et anciennement \emph{Mare puteolanum} ; et en effet Pline vante les pourpres de ce pays, \emph{P. puteolanæ}. Elle fournit une si grande quantité de liqueur pourprée, que Fabius Columna dit que, lorsqu'on la sert sur la table, quoique cuite, il l'a vue verser pour ainsi dire la pourpre et en tacher les doigts et les linges de table. On mange donc encore cet animal comme on le faisait autrefois ; et cependant on n'emploie plus la pourpre qu'il fournit pour la teinture ; non pas, ajoute Columna, seulement par ignorance et à cause de la grande dépense et la difficulté, mais à cause de la grande abondance du fucus, nommé \emph{rocella}, dont les teinturiers extraient d'excellente pourpre avec beaucoup moins de frais et de difficultés, et par conséquent avec bien plus de bénéfice.

Malgré ces observations de Fabius Columna, aucun des auteurs subséquents n'a regardé la coquille qu'il a décrite comme celle qui fournissait la pourpre des anciens.

En 1685, Guill. Cole donna des détails intéressants sur une espèce de coquillages qu'il avait trouvés, en grande abondance, sur la côte d'Irlande et qui fournit une véritable liqueur pourpre. On trouve dans le Journal des savants pour l'année 1686, un extrait, à ce qu'il me semble, de ce que Cole venait de publier sur les changements de couleur de l'humeur colorée de la pourpre. On reconnaît aisément qu'il est question de l'animal que M. de Lamarck à nommé \emph{purpura lapillus}, \emph{B. lapillus} de Linné : elle ne paraît donc avoir aucun rapport avec la pourpre des anciens, à moins que de croire que ce serait le \emph{buccinum} de Pline ; mais c'est à cet auteur que l'on doit les premières observations sur les changements qu'éprouve la matière colorante avant de devenir d'un rouge pourpre très-foncé.

Lister, en 1693, confirma l'observation de Cole, et montra, par un passage de Bede, dans son Histoire ecclésiastique, que cette espèce de teinture, qu'on croyait propre aux Tyriens, aux Grecs et aux Romains, se faisait aussi en Angleterre, où elle était également fort estimée.

Réaumur (Mémoires de l'Académie des sciences, année 1711) confirma par de nouvelles observations que le même coquillage décrit par Cole, donnait une pourpre très-brillante ; mais n'ajouta réellement rien à ce qu'avait dit le naturaliste anglais, dont il paraît même n'avoir pas connu le travail, puisqu'il ne le cite pas.

Tempelmann, dans une Dissertation sur la pourpre des anciens, insérée dans le Magasin de Décembre 1753, décrit la même espèce que Cole et Réaumur, et cependant il a l'air de croire que ce n'est pas le même animal, qu'il dit n'avoir pas vu sur les côtes d'Angleterre ; du reste il confirme les observations des auteurs précités.

Depuis ce temps il paraît que Stroëm a aussi fait quelques recherches sur la liqueur fournie par le même mollusque.

Malgré tout cela, plusieurs auteurs ont regardé comme fournissant la pourpre des anciens, des coquilles du genre \emph{Turbo}, et entre autres les \emph{turbo scalata} et \emph{clathrus} de Linné ; mais certainement à tort, car ces animaux n'ont pas de liqueur colorée.

M. G. Cuvier a été probablement beaucoup plus près de la vérité, en admettant avec Rondelet que c'est le \emph{murex brandaris} de Linné. En effet, cet animal est assez commun dans la Méditerranée. Il est certain qu'il donne une humeur colorante, et sa forme convient assez bien à la description d'Aristote et de Pline.

M. Bory de Saint-Vincent n'ayant pu vraisemblablement réussir à trouver le coquillage qui fournissait la pourpre des anciens, a cherché à établir que les Phéniciens faisaient la pourpre avec l'orseille, \emph{lichen rocella}, Linn., et que c'était pour donner le change qu'ils disaient la tirer d'un coquillage. Mais, comme le fait justement observer M. Bosc, les passages d'Aristote et de Pline sont trop formels pour laisser aucun doute à ce sujet.

Il ne paraît pas que les habitants de la Méditerranée emploient davantage la pourpre, que l'on ne le faisait du temps de F. Columna ; mais il se pourrait qu'on s'en servit encore pour marquer le linge dans quelques endroits des côtes d'Angleterre ou d'Irlande, comme cela avait lieu du temps de Cole et même de Lister. Quoi qu'il en soit, voici les phénomènes que présente la matière colorante aussitôt qu'elle a été appliquée à l'aide d'un petit pinceau sur le linge, la laine ou même la soie, sans aucune préparation préliminaire. La couleur primitive est d'un blanc jaunâtre ou légèrement verdâtre, c'est-à-dire celle du pus d'un ulcère ; immédiatement après ce vert clair, elle paraît d'un vert foncé, qui en peu de minutes se change en vert de mer ; au bout de quelques autres minutes celui-ci tourne au bleu pâle, qui, peu de temps après, devient rouge purpurin, et enfin dans l'espace d'une heure ou deux, cette couleur devient d'un rouge-pourpre très-foncé. Mais, pour pouvoir apercevoir ces nuances successives d'une manière tranchée, il faut exposer l'étoffe à l'action solaire, en ayant soin, en été, de ne le faire qu'une heure ou deux après le lever ou avant le coucher du soleil ; car dans le milieu du jour la couleur changerait si promptement, qu'il serait impossible d'apercevoir les nuances intermédiaires. En hiver, le soleil de midi les laisse très-bien saisir. Pendant l'exposition au soleil, l'étoffe teinte exhale une odeur fétide, très-forte, que Cole compare à celle qui s'exhale d'un mélange d'ail et d'assa fœtida. La chaleur du feu produit les mêmes effets que les rayons solaires, mais beaucoup plus lentement ; et Réaumur a fait l'observation qu'il faut que la chaleur du feu soit beaucoup plus grande que celle du soleil pour produire les mêmes changements. le même observateur a vu que l'on peut aussi les obtenir à l'ombre, ou mieux à la lumière diffuse, en exposant l'étoffe à l'air seulement ; mais alors le passage du blanc verdâtre au pourpre violet est beaucoup plus lent, à moins qu'il ne fasse un très-grand vent ; car dans ce cas le changement se fait aussi rapidement que si l'étoffe était exposée aux rayons modérés du soleil. D'après cela, on pourrait conclure que l'air et la chaleur sont nécessaires dans cette espèce de purpurification. C'est ce qui n'est réellement pas, du moins d'après les observations nombreuses et contradictoires de Duhamel de Monceau, imprimées dans les Mémoires de l'Académie des sciences pour 1736, et faites, il est vrai, sur une autre espèce de mollusques que celle qui a servi aux expériences de Cole et de Réaumur, peut-être sur la même que F. Columna a donné pour celle des anciens. En effet, suivant Duhamel, un linge frotté de la liqueur fournie par le coquillage, et dont une partie seulement est exposée au soleil, ne rougit que dans cette partie ; ce qui ne devient pas pourpre ou rouge, reste vert : un soleil plus fort rend les changements de couleur plus prompts, et peut-être aussi les couleurs plus vives ; c'est surtout ce qu'on voit très-bien en employant une lentille ou le foyer d'un miroir ardent. Si sur un linge frotté de ce suc, et exposé au soleil, on met un petit corps opaque comme une pièce de monnaie ou une pièce de cuivre très-mince, l'étoffe rougit partout, si ce n'est dans l'endroit recouvert par le corps. Si l'on emploie, au contraire, un corps transparent, comme du verre, fût-il épais de trois doigts, la purpurification a complétement lieu : si l'on emploie trois morceaux de papier, l'un noirci avec de l'encre, l'autre dans son état naturel, et le troisième huilé, la coloration en rouge est proportionnelle à leur degré de transparence ; sous des papiers également opaques, mais différemment colorés ; il y a plus de coloration en rouge sous le bleu, et moins sous le rouge que sous les autres. La chaleur du feu, celle du fer rouge, ne produisent aucun changement de couleur en rouge : elle devient verte et ensuite jaune ; mais l'étoffe, qui dans ces expériences ne s'est colorée qu'en vert, devient rouge, immédiatement aussitôt qu'elle est frappée par un rayon de soleil, même passant par une fente étroite. L'acide sulfureux, produit par la combustion du soufre, ne rougit pas non plus l'étoffe imprégnée de matière colorante de la pourpre. Ce changement de couleur n'est pas dû à une évaporation, comme on aurait pu le croire par l'odeur fétide qui s'exhale de l'étoffe imprégnée ; car, mise entre deux verres bien serrés et exposée au soleil, une couleur rouge très-vive s'est produite presque instantanément. Cependant dans les mois de Janvier et de Février on n'a pas observé tout-à-fait le même effet que dans le mois de Mars, où la chaleur est déjà d'une assez grande force en Provence et où les expériences ont été faites. Dans les mois plus chauds, l'air, bien échauffé, même pendant les temps couverts, suffit pour produire une couleur rouge, tandis que dans l'hiver le soleil seul à cet effet ; d'où Duhamel conclut que la lumière et la chaleur du soleil agissent ensemble et séparément : mais que la lumière est toujours assez forte pour agir, et agit beaucoup plus, tandis que la chaleur a besoin d'être à un certain degré. Il paraît même que les différents rayons du spectre solaire n'exercent aucune action dans ce changement de couleur, et qu'il faut qu'ils soient réunis pour cela. Le rayon rouge semble cependant avoir une action un peu plus forte que les autres. La lumière de la lune, quoique très-concentrée au moyen d'une lentille, ne produit aucun effet, non plus que celle qui provient des bougies.

Mais si les observateurs ne sont pas tout-à-fait d'accord sur les circonstances qui font passer la matière colorante de la pourpre du blanc verdâtre au rouge pourpre, il n'en est pas de même sur la fixité de cette teinture. Cole, Réaumur, Tempelmann, et surtout Duhamel, ont prouvé que, lorsque l'étoffe a été parfaitement imbibée de la matière, et que toutes ses parties ont été complétement exposées à l'action solaire, les lessives les plus fortes, les débouillis les plus actifs, n'ont aucune action sur la couleur, si ce n'est quand il en reste plusieurs couches à la surface, la dernière, ayant empêché l'action solaire sur les autres, et sa combinaison avec le tissu n'ayant pas eu lieu ; alors la couleur s'éclaircit beaucoup, en sorte que Duhamel conclut de ses expériences à ce sujet, qu'il fallait que les anciens eussent un procédé particulier pour étendre la matière colorante, toujours assez épaisse et visqueuse sur l'animal, et ainsi la faire pénétrer dans toutes les parties du tissu ; peut-être était-ce à quoi servait l'eau, l'urine et le sel, que les anciens teinturiers en pourpre employaient comme nous l'apprenons de Pline. Tempelmann dit cependant, que l'expérience lui a prouvé que l'addition du sel ne sert à rien. C'est un sujet que les chimistes auraient pu éclaircir ; malheureusement nous n'en connaissons aucun qui s'en soit occupé, ce qui aurait été cependant avantageux ; sinon pour l'art de la teinture, puisqu'il paraît que la couleur pourpre des modernes est aussi belle, aussi fixe, et bien plus facile à obtenir, et par conséquent bien moins coûteuse que celle des anciens, mais du moins pour la science de la chimie animale. Duhamel termine ses expériences en disant qu'il pense que l'action du soleil dans la purpurification a quelque chose d'analogue à ce qui se passe dans la coloration des fruits, qui restent blanchâtres, jaunes ou verts dans les endroits ombragés, et qui ne se colorent que dans ceux qui reçoivent l'action du soleil ; mais ici ce changement se fait peu à peu, très-lentement, tandis que dans la pourpre il a lieu presque instantanément.

Comme conclusions de cet article, nous voyons :
\begin{enumerate}
    \item Il est probable que le mollusque dont les anciens tiraient principalement leur pourpre, est un animal assez gros, connu dans la Méditerranée, parfaitement décrit par F. Columna, et dont Linné et les conchyliologistes modernes font leur \emph{murex trunculus}, ou peut-être le \emph{murex brandaris}.

    \item Ils employaient aussi une espèce de buccin plus petite, pour obtenir une couleur analogue, mais un peu différente, et cette espèce n'est probablement pas le véritable \emph{B. lapillus} de Linné.

    \item Il est certain qu'un assez grand nombre d'espèces de la famille des siphonobranches fournissent une liqueur analogue ; mais il est probable qu'elles n'en produisent pas toutes, du moins Duhamel le dit positivement.

    \item Il est même possible que tous les individus d'une même espèce n'en fournissent pas. Cela dépend-il du sexe, de l'âge, de l'époque de la reproduction ? c'est ce que nous ignorons.

    \item Nous ne savons pas davantage au juste dans quelle partie de l'animal se trouve cette matière : est-ce dans l'organe dépurateur ? est-ce dans l'appareil générateur lui-même ? Ce qui pourrait porter à le croire, c'est que les œufs du \emph{B. lapillus} contiennent la même liqueur en abondance, comme l'a observé Réaumur. Et alors on pourrait penser qu'il ne s'en trouve que dans les femelles, ce qui expliquerait l'observation de Duhamel, qui dit avoir vu des individus de la même espèce en avoir, et d'autres n'en avoir pas.

    \item Le procédé de la teinture des anciens est encore inconnu.

    \item Les phénomènes chimiques de la purpurification ne le sont encore que très-incomplètement.
\end{enumerate}
\paragraph{}
Passons maintenant à la distinction des espèces.

\bigskip

La \textbf{P. persique} : \emph{P. persica}, de Lamk. ; \emph{Buccinum haustorium}, Linn., Gmel., p. 3498, n.° 175 ; Enc. méth., pl. 397, fig. 1, \emph{a}, \emph{b}, vulgairement là \textbf{Conque persique}. Coquille assez grande, ovale, sillonnée transversalement ; spire courte ; ouverture très-évasée, la columelle étant excavée longitudinalement dans sa longueur ; le bord droit, sillonné en dedans ; couleur d'un brun noirâtre, maculé de blanc sur les sillons ; columelle jaune ; l'intérieur du bord droit noirâtre.

De l'Océan des grandes Indes, et des rivages de la Nouvelle-Zélande, suivant Gmelin.

\bigskip

La \textbf{Pourpre tachetée} ; \emph{P. Rudolphi}, Chemn., \emph{Conch.}, 10, t. 154, fig. 1467, 1468. Coquille ovale, sillonnée en travers, à spire un peu plus élevée que dans la précédente, et à tours de spire noduleux et anguleux à leur bord supérieur ; l'ouverture est aussi moins dilatée, non rayée dans le fond, et la columelle plus étroite. Couleur d'un brun noirâtre, marqué de grosses taches noires et blanches, outre celles articulées des sillons.

De l'Océan des grandes Indes, comme la précédente, dont elle n'est sans doute qu'une variété de sexe.

\bigskip

La \textbf{P. antique} : \emph{P. patula}, Linn., Gmel., pag. 3483, n.° 51 ; Martini, \emph{Conch.}, 3, tab. 69, fig. 758, 759. Coquille ovale, sillonnée en travers, hérissée de tubercules, surtout dans le jeune âge ; à spire assez courte ; l'ouverture évasée. Couleur d'un roux noirâtre en dehors ; la columelle d'un jaune roussâtre ; le bord droit blanc.

De l'Océan atlantique et de la Méditerranée, où elle est assez commune pour que F. Columna ait pensé que c'était de cette espèce que les Romains tiraient leur couleur pourpre.

\bigskip

La \textbf{P. columellaire} : \emph{P. columellaris}, de Lamk., Anim. sans vert.. tom. 7, pag. 236, n.° 4 ; Enc. méth., pl. 398, fig. 3, \emph{a}, \emph{b}. Coquille ovale, épaisse, striée et rugueuse en travers ; spire courte ; columelle plane avec un pli au milieu ; le bord droit garni à l'intérieur d'une série de dents assez fortes. Couleur roussâtre.

Patrie inconnue.

Est-ce bien une véritable pourpre.

\bigskip

La \textbf{P. cordelée} : \emph{P. succincta}, de Lamk. ; \emph{B. orbitus}, Linn., Gmel., p. 3490, n.° 183 ; Enc. méth., pl. 398, fig. 1, \emph{a}, \emph{b}. Coquille assez épaisse, ovale, striée en travers, et comme cordelée par de grosses rugosités épaisses, élevées, costiformes ; la partie supérieure des tours de spire se relevant au-dessus de la suture. Couleur grise.

Des mers de la Nouvelle-Zélande.

\bigskip

La \textbf{Pourpre consul} : \emph{P. consul}, de Lamk. ; \emph{Murex consul}, Linn., Gmel., p. 3540, n.° 159 ; Chemn., \emph{Conch.}, 10, tab. 160, fig. 1516, 1517. Très-grande coquille, épaisse, pesante, ovale, turbinée, ventrue ; à spire conique, aiguë, garnie au bord supérieur de ses tours de tubercules noduleux, qui, sur le dernier, sont très-grands et comprimés ; le bord droit sillonné en dedans et échancré en arrière. Couleur blanche.

De l'Océan indien.

\bigskip

La \textbf{P. armigère} ; \emph{P. armigera}, \emph{Bucc. armigerum}, Chemn., \emph{Conch.}, 11, tab. 187, fig. 1798, 1799. Coquille ovale, subturbinée, striée en travers ; spire conique, garnie de plusieurs rangs de tubercules noduleux, obtus, alongés, dont ceux des deux rangées supérieures du dernier tour sont plus grands ; columelle avec trois plis peu marqués en avant ; bord droit, mince et sinueux. Couleur blanc-jaunâtre.

Patrie inconnue.

\bigskip

La \textbf{P. bituberculaire} : \emph{P. bitubercularis}, de Lamk., \emph{loc. cit.}, p. 137, n.° 8 ; Séba, \emph{Mus.}, 3, tab. 52, fig. 22, 23. Coquille ovale, à spire un peu élevée, hérissée de tubercules aigus sur deux rangs, aux deux derniers tours ; ouverture lisse. Couleur peinte longitudinalement de noir et de blanc ; les tubercules noirs.

Patrie inconnue.

\bigskip

La \textbf{P. marron d'Inde} : \emph{P. hippocastanum}, \emph{Mur. hippocastanus} ; Linn., Gmel., p. 3539, n.° 48 ; Martini, \emph{Conch.}, 3, tab. 99, fig. 945, 946. Coquille ovale, hérissée de tubercules alongés, spiniformes, et cerclée de sillons subsquameux ; bord droit, sinueux et verruqueux en dedans. Couleur marbrée de noir et de blanc.

De l'Océan des grandes Indes.

\bigskip

La \textbf{P. ondée} : \emph{P. undata}, de Lamk., \emph{loc. cit.}, pag. 238, fig. 10 ; Chemn., \emph{Conch.}, 11, tab. 192, fig. 1861, 1862. Coquille ovale, aiguë , très-finement striée en travers ; tours de spire un peu anguleux à leur partie supérieure, et hérissés de petits tubercules aigus sur deux rangs au dernier. Couleur peinte de blanc et de brun noirâtre en ondes longitudinales ; ouverture blanche, avec quelques sillons en dedans du bord droit.

Patrie inconnue.

\bigskip

La \textbf{Pourpre hémastome} : \emph{P. hæmastoma}, \emph{B. hæmastoma}, Linn., Gmel., pag. 3483, n.° 52 ; Martini, \emph{Conch.}, 3, tab. 101, fig. 964, 965. Coquille un peu épaisse, ovale, conique, striée en travers ; tours de spire obtusément anguleux à leur bord supérieur, et garnis de nodules sur quatre rangs au dernier. Couleur d'un fauve roussâtre en dehors ; l'ouverture jaune pourprée.

De l'Océan atlantique, peut-être de celui des grandes Indes, suivant M. de Lamarck. Gmelin l'indique aussi comme de la Méditerranée.

\bigskip

La \textbf{P. bourgeonnée} : \emph{P. mancinella}, \emph{Mur. mancinella}, Linn., Gmel., pag. 3538, n.° 47 ; \emph{Purp. gemmulata}, de Lamk., Enc. méth., pl. 397, fig. 3, \emph{a}, \emph{b}. Coquille ovale, ventrue, épaisse ; spire conique, aiguë, hérissée de tubercules subaigus, disposés en rangées transversales ; columelle striée ; couleur blancrougeâtre ; la base des tubercules rouges ; l'ouverture jaune ; les stries du bord droit rouges.

Des mers de l'Inde, suivant M. de Lamarck, et en outre de l'Afrique occidentale, suivant Gmelin.

M. de Lamarck regarde comme une variété de cette espèce, une coquille plus petite, oblongue, d'un blanc jaunâtre, et dont les tubercules gemmiformes sont orangés. Elle est figurée dans de Born, \emph{Mus.}, t. 9, fig. 19 et 20.

\bigskip

La \textbf{P. crapaud} : \emph{P. bufo}, de Lamk., \emph{loc. cit.}, p. 239, n.° 13 ; Petiv., Gaz., tab. 19, fig. 10. Coquille ovale, raccourcie, ventrue, striée en travers ; spire très-courte, un peu aiguë, tuberculifère ; les tubercules sur quatre séries au dernier tour ; ouverture dilatée, très-lisse, d'un blanc jaunâtre.

Des mers de l'Inde ?

\bigskip

La \textbf{P. calleuse} : \emph{P. callosa}, de Lamk., p. 239, n.° 14 ; Séba, \emph{Mus.}, 3, t. 60, fig. 11 ; vulgairement le \textbf{Cul-de-singe}. Coquille obovale, ventrue, striée en travers ; spire très-courte , comme écrasée, calleuse, mucronée, tuberculifère ; les tubercules sur deux rangs au dernier tour ; couleur d'un gris brunâtre en dehors ; l'ouverture blanc-jaunâtre très-lisse.

Patrie inconnue.

\bigskip

La \textbf{Pourpre néritoïde} : \emph{P. neritoides}, de Lamk. ; \emph{Mur. fucus}, Gmel., p. 3538, n.° 44 ; Mart., \emph{Conch.}, 3, t. 100, fig. 959-962. Coquille épaisse, ovale, raccourcie, ventrue, striée en travers, à spire très-courte, comme écrasée, garnie de tubercules, en quatre séries sur le dernier tour ; columelle plane, très-large, biponctuée de noir : couleur d'un blanc sale.

Patrie inconnue.

\bigskip

La \textbf{P. planospire} : \emph{P. planospira}, de Lamk., p. 240, n.° 16 ; \emph{P. lineata}, Enc. méth., pl. 397, fig. 5, \emph{a}, \emph{b}. Coquille épaisse, subhémisphérique, ventrue, cerclée de côtes distantes, subaiguës, à spire comme tronquée, plane et même un peu enfoncée ; columelle profondément excavée dans son milieu ; bord droit, épais ; couleur générale blanche, rayée de jaune en dehors et d'un orangé rougeâtre très-vif en dedans.

Patrie inconnue.

\bigskip

La \textbf{P. callifère} ; \emph{P. callifera}, de Lamk., 240, 17. Coquille ventrue, semi-globuleuse, à spire courte, mamillaire au sommet et comme couronnée par une rangée de callosités gibbeuses, décurrentes ; ouverture lisse : couleur blanchâtre.

Patrie inconnue.

\bigskip

La \textbf{P. couronnée} : \emph{P. coronata}, de Lamk., 241, 18 ; le \textbf{Labarin}, Adans., Sénég., pl. 7, fig. 2 ; Enc. méth., pl. 397, fig. 4. Coquille ovale, aiguë, ventrue, striée grossièrement en travers, à spire conique, couronnée par une série décurrente de tubercules plus alongés et plus droits sur le dernier tour, et à suture imbriquée et laciniée : couleur brun-noirâtre en dessus, cendrée intérieurement ; l'ouverture jaunâtre.

Des mers du Sénégal.

\bigskip

La \textbf{P. carinifère} : \emph{P. carinifera}, de Lamk., 241, 19 ; Mart., \emph{Conch.}, 3, t. 100, fig. 951 ? Coquille ovale, aiguë, à tours de spire striés, très-anguleux ou carenés ; la carène hérissée de tubercules distants et doublés sur le dernier ; ouverture lisse : couleur fauve roussâtre.

De l'Océan atlantique austral ?

\bigskip

La \textbf{P. escalier} ; \emph{P. scalariformis}, de Lamk., 241, 20. Coquille ovale, scalariforme, ombiliquée, à spire saillante ; les tours treillissés, aplatis en dessus et carenés ; ouverture ronde ; le bord droit, sillonné en dedans : couleur blanche.

Patrie inconnue.

\bigskip

La \textbf{Pourpre pagode} : \emph{P. sacellum}, \emph{Mur. sacellum}, Linn., Gmel., p. 3530, n.° 164 ; Chemn., \emph{Conch.}, 10, tab. 163, fig. 1561 et 1562. Coquille ovale, scalariforme, ombiliquée, à tours de spire striés et cordonnés, aplatis en dessus, et muriqués le long d'une carène décurrente : ouverture arrondie, ovale, à bord droit, crénelé et sillonné en dedans.

Cette espèce, qui vient de la mer des Indes, près les îles de Nicobar, n'appartient pas à ce genre, s'il est vrai, comme le dit Gmelin, qu'elle ait un tube droit un peu relevé.

\bigskip

La \textbf{P. écailleuse} : \emph{P. squamosa}, de Lamk., 7, 242, 22 ; Enc. méth., pl. 398, fig. 2, \emph{a}, \emph{b}. Coquille ovale, aiguë , à tours de spire séparés par une suture étranglée, convexes, subtreillissés par ces stries décurrentes très-fines, et par des sillons transverses, aigus, comme écailleux ; bord droit, denticulé : couleur jaune, testacée en dehors, blanche en dedans.

\bigskip

La \textbf{P. ridée} : \emph{P. rugosa}, de Lamk. ; \emph{Buccinum bicostatum}, Brug., Encycl., Dict. n.° 17, et \emph{B. lacunosum}, n.° 19 ; Chemn., \emph{Conch.}, 10, t. 154, fig. 1473. Coquille ovale-oblongue, à tours de spire convexes, rendus rugueux par des sortes de rides décurrentes, alternativement grandes et petites, et légèrement imbriquées d'écailles ; le bord externe sillonné intérieurement : couleur d'un blanc sale à l'état adulte, et avec quelques teintes brunes dans le jeune âge.

Des mers de la Nouvelle-Zélande.

\bigskip

La \textbf{P. nattée} : \emph{P. textilosa}, de Lamk., 7, 242, 24 ; Enc. méth., pl. 398, fig. 4, \emph{a}, \emph{b}. Coquille ovale, aiguë, un peu ventrue, à spire médiocre ; les tours treillissés par des stries d'accroissement très-fines et des grosses rides décurrentes non écailleuses, alternativement grandes et petites ; ouverture évasée ; le bord externe profondément sillonné intérieurement : couleur d'un blanc sale.

Des mers de la Nouvelle-Hollande.

\bigskip

La \textbf{P. guirlande} : \emph{P. coronatum} ; \emph{B. coronatum}, Linn., Gmel., 3486, n.° 86 ; \emph{B. sertum}, Brug., Dict., n.° 25 ; \emph{P. sertum}, de Lamk., Enc. méth., pl. 397, fig. 2. Coquille ovale-oblongue, à tours de spire convexes, aplatis en dessus et treillissés par des stries d'accroissement imprimés, et par des stries décurrentes granuleuses ; ouverture avec un sinus postérieur formé par une dent de chaque bord ; le droit lisse : couleur variée de grandes taches blanches et rousses, inégales ; la columelle fauve.

Patrie inconnue.

\bigskip

La \textbf{Pourpre Francolin} : \emph{P. Francolinus}, \emph{B. Francolinus}, Brug., Dict., n.° 24 ; Séba, \emph{Mus.}, 3, t. 53, fig. \emph{T}. Coquille ovale oblongue, assez lisse, à tours de spire convexes, aplatis en dessus et treillissés par des stries d'accroissement plus fines et par des stries décurrentes également moins marquées et nullement granuleuses : couleur fauve roussâtre, ornée de petites taches blanches épaisses ; sinus postérieur comme dans la précédente.

Cette espèce, dont on ignore la patrie, doit-elle être distinguée de la P. guirlande ?

\bigskip

La \textbf{P. a collet} ; \emph{P. limbosa}, de Lamk., 7, 243, 27. Coquille ovale-oblongue, à tours de spire aplatis en avant de la suture et très-finement striés dans leur décurrence ; bord droit mince : couleur d'un fauve rougeâtre.

Patrie inconnue.

\bigskip

La \textbf{P. ficelée} ; \emph{P. ligata}, de Lamk., 7, 244, 28. Coquille ovale-oblongue, à tours de spire convexes, aplatis à leur partie postérieure, cerclés de rugosités un peu convexes : couleur d'un gris roussâtre en dehors, blanche et lisse en dedans.

Patrie inconnue.

\bigskip

La \textbf{P. fustigée} : \emph{P. cruenta}, \emph{B. cruentatum}, Linn., Gmel., p. 3491, n.° 88 ; Martini, \emph{Conch.}, 4, t. 123, fig. 1143 et 1144. Coquille ovale, aiguë, à tours de spire convexes, subanguleux, cerclés de stries extrêmement fines : couleur grise, parsemée de taches irrégulières rouges ou ventre de biche en dehors ; l'ouverture d'un jaune testacé ; le bord droit strié en dedans.

Des mers de la Guiane.

\bigskip

La \textbf{P. a teinture} : \emph{P. lapillus}, \emph{B. lapillus}, Linn., Gmel., p. 3484, n.° 53 ; Martini, \emph{Conch.}, 3, t. 1121, fig. 1111 et 1112, et 4, t. 1122, fig. 1128 et 1129. Coquille épaisse, ovale, aiguë, à spire conique ; les tours convexes, plus ou moins fortement striés, suivant la décurrence de la spire ; bord droit épais et denticulé en dedans ; couleur extrêmement variable, mais le plus ordinairement d'un blanc sale ou jaunâtre, zoné de brun plus ou moins foncé.

Cette espèce, fort commune sur les bords de la Manche, sur les côtes de l'Océan et même sur celles de l'Afrique occidentale, puisque le libot d'Adanson (Sénég., pl. 7, fig. 4) lui appartient, présente un grand nombre de variétés, d'abord dans la taille, ensuite dans l'état lisse ou rugueux, et, enfin, dans la couleur ; ainsi, il y en a de toutes blanches, de toutes brunes et de toutes jaunes ; le plus grand nombre est lisse ou peu rugueux ; mais on en trouve aussi qui offrent des côtes assez bien marquées et hérissées d'écailles imbriquées. M. de Lamarck a cru devoir former de cette variété une espèce distincte, qu'il a nommée la \textbf{P. imbriquée}, \emph{P. imbricata}. Ayant eu l'occasion de voir cette coquille sur les côtes de la Manche et d'en observer l'animal, nous pouvons assurer que ce n'est qu'une simple variété du \emph{P. lapillus}.

Réaumur (Mémoires de l'Académie des sciences de Paris, année 1711), a donné des détails curieux sur cette espèce, pour prouver qu'elle pouvait fournir de la pourpre. Ses œufs, graniformes, déposés un à un sur les rochers, ont été longtemps regardés comme une espèce d'hydre, \emph{H. triticea}. Cette erreur a été relevée par Stroëm, qui a beaucoup étudié ce mollusque.

\bigskip

La \textbf{Pourpre calebasse} : \emph{P. lagenaria}, de Lamk., 7, 245, 32 ; Rumph., \emph{Mus.}, t. 24, fig. \emph{D}. Coquille ovale, à spire courte, un peu obtuse, à tours anguleux en arrière, plats, comprimés au-dessous de la suture, très-finement striés en travers ; bord droit mince, lisse en dedans : couleur fauve, coupée de bandes blanches et ornée de petites lignes longitudinales ondées, ventre de biche en dehors, fauve rougeâtre en dedans.

Patrie inconnue.

La \textbf{P. cataracte} : \emph{P. cataracta}, \emph{B. cataracta}, Linn., Gmel., pag. 3498, fig. 177 ; Chemn., \emph{Conch.}, 10, t. 152, fig. 1455. Coquille ovale, aiguë, à tours de spire subanguleux en arrière, treillissés par des stries décurrentes un peu saillantes, et des stries d'accroissement imprimées ; bord externe strié en dedans ; couleur grise, ornée de bandes longitudinales ondées, brunes.

Des mers de la Nouvelle-Zélande.

\bigskip

La \textbf{P. bicostale} : \emph{P. bicostalis}, de Lamk., 7, 243, 34 ; Enc. méth., pl. 398, fig. 5, \emph{a}, \emph{b}. Coquille ovale, aiguë ; à tours de spire striés dans leur décurrence, anguleux en arrière et garnis de tubercules en double série sur le dernier ; le bord droit sillonné à l'intérieur : couleur grise, peinte de bandes longitudinales, flexueuses, d'un brun roux en dehors.

Patrie inconnue.

\bigskip

La \textbf{Pourpre plissée} : \emph{P. plicata}, \emph{Murex plicatus}, Linn., Gmel., p. 3551, n.° 94 ; Martini, \emph{Conch.}, 4, t. 123, fig. 1141 et 1142. Coquille conique, à spire courte, obtuse au sommet , plissée obliquement et dans la longueur, hérissée de tubercules, formant quatre séries sur le dernier tour ; bord droit denté en dedans ; couleur noire et blanche, par bandes longitudinales.

De l'Océan indien ?

\bigskip

La \textbf{P. corbulée} : \emph{P. fiscella}, \emph{M. fiscellus}, Linn., Gmel., pag. 3552 ; Chemn., \emph{Conch.}, 10, t. 160, fig. 1524 et 1525. Coquille ovale-oblongue, à spire un peu saillante, assez obtuse, plissée et nodulée dans sa longueur, striée en travers ; ouverture peu évasée ; couleur noire et blanche en dehors, teintée de rose violâtre en dedans.

Des mers de la Chine.

\bigskip

La \textbf{P. thiarelle} : \emph{P. thiarella}, de Lamk., 7, 246, 37. Assez petite coquille ovale, aiguë, un peu ventrue, striée, à tours de spire anguleux, aplatis en arrière et couronnés par une série décurrente de tubercules ; le bord droit sillonné en dedans : couleur d'un gris fauve.

Patrie inconnue.

\bigskip

La \textbf{P. rustique} ; \emph{P. rustica}, de Lamk., 7, 246, 38. Petite coquille ovale, aiguë, striée en travers, plissée et noduleuse dans sa longueur ; tours de la spire anguleux : couleur variée de brun sur les plis, plombée dans les intervalles et jaunâtre sur les nodules.

Patrie inconnue.

\bigskip

La \textbf{P. semi-imriquée} ; \emph{P. semi-imbricata}, de Lamk., 7, 246, 59. Coquille ovale, aiguë, à spire saillante, côtelée dans sa décurrence ; les côtes du dernier tour imbriquées d'écailles ; ouverture oblongue, un peu resserrée en arrière ; bord droit épais, denticulé en dedans : couleur blanche.

Des côtes occidentales du Mexique.

\bigskip

La \textbf{P. échinulée} ; \emph{P. echinulata}, de Lamk., 7, 247, 40. Coquille ovale, ventrue, striée très-finement en travers, et plissée dans sa longueur ; spire courte, obtuse ; tours anguleux en arrière et hérissés de tubercules nombreux, sur quatre rangs au dernier ; ouverture lisse ; le bord externe jaunâtre en dedans.

Patrie inconnue.

M. de Lamarck avait cru que ce pourrait être le \emph{M. mancinella} de Linné ; mais il paraît qu'elle en diffère par plusieurs caractères.

\bigskip

La \textbf{Pourpre hérisson} : \emph{P. hystrix}, \emph{M. hystrix}, Linn., Gmel., p. 3538, n.° 46 ; Martini, \emph{Conch.}, 3, t. 101, fig. 974 et 975. Coquille ovale, ventrue, striée en travers, à spire courte, aiguë, hérissée d'épines assez longues, canaliculées, sur quatre séries décurrentes ; ouverture grande ; bord droit denticulé intérieurement ; columelle un peu ridée en avant : couleur jaunâtre en dehors, un peu rosée en dedans ; mais quelquefois variée de brun et de blanc, unicolore ou tachetée.

Patrie inconnue.

\bigskip

La \textbf{P. deltoïde} ; \emph{P. deltoidea}, de Lamk., 7, 247, 42. Coquille ovale, raccourcie, ventrue, subdeltoïde ; à spire courte, un peu obtuse ; le dernier tour couronné en arrière de tubercules rares, assez grands, et de nodosités en avant ; le bord droit lisse ; couleur rougeâtre.

Patrie inconnue.

\bigskip

La \textbf{P. unifasciale} : \emph{P. unifascialis}, de Lamk., 7, 247, 43 ; Enc. méth., pl. 397, fig. 6. Coquille mince, ovale, aiguë, ventrue, très-finement striée en travers, à spire courte ; le dernier tour garni postérieurement d'une série de nodules striés transversalement ; ouverture très-large, à bord droit mince, strié en dedans : couleur roussâtre, avec une bandelette blanche en travers.

Patrie inconnue.

\bigskip

La \textbf{P. rétuse} : \emph{P. retusa}, de Lamk., 7, 248, 44 ; \emph{Buccinum fossile}, Linn., Gmel., p. 3485, n.° 58 ? Mart., \emph{Conch.}, 3, t. 94, fig. 912. Coquille ovale, lisse, à spire très-courte, rétuse ; le dernier tour subanguleux au milieu, puis excavé et renflé à sa partie postérieure ; ouverture petite, lisse ; la columelle renflée et calleuse en arrière, arquée en avant ; le bord droit mince : couleur d'un blanc sale.

Patrie inconnue.

M. de Lamarck pense que cette coquille, qui faisait partie de son cabinet, ne correspond pas exactement au \emph{B. fossile} de Martini, et en effet elle n'est pas fossile, comme celui-ci.

\bigskip

La \textbf{Pourpre cabestan} : \emph{P. trochlea}, \emph{B. scala}, Linn., Gmel., p. 3485, n.° 61 ; \emph{Triton cochlea}, de Lamk., Enc. méth., pl. 422, fig. 4, \emph{a}, \emph{b}. Coquille ovale, à spire assez saillante, cerclée dans toute sa décurrence par des bourrelets élevés, assez larges, très-lisses, au nombre de trois sur le dernier tour et formant trois saillies sur le bord droit, constamment mince : couleur cendrée.

Du détroit de Magellan et des mers du cap de Bonne-Espérance.

C'est une coquille rare et fort recherchée.

\bigskip

La \textbf{P. cheville} ; \emph{P. clavus}, de Lamk., 7, 248, 46. Coquille ovale, conique, grêle, presque turriculée, scalariforme, à sommet aigu, striée très-élégamment en travers, côtelée dans sa longueur ; bord externe mince et strié en dedans : couleur d'un gris bleuâtre en dehors, rougeâtre en dedans.

Patrie inconnue.

\bigskip

La \textbf{P. fasciolaire} : \emph{P. fasciolaris}, de Lamk., 7, 249, 47 ; Gualt., \emph{Test.}, tab. 65, fig. \emph{G} ? Coquille ovale, conique, luisante, très-finement striée en travers, avec un seul pli à la partie postérieure de la columelle, et le bord droit strié en dedans : couleur d'un blanc bleuâtre, mêlé de fauve, avec des fascies nombreuses, articulées de blanc et de brun.

Patrie inconnue.

\bigskip

La \textbf{P. pavillon} : \emph{P. vexillum}, \emph{Strombus vexillum}, Gmel., pag. 3520, n.° 52 ; Chemn., \emph{Conch.}, 10, t. 157, fig. 1504-1506. Coquille petite, lisse, luisante, ovale, subcylindrique, à spire courte, obtuse ; ouverture évasée à sa base, avec un canal très-court, ou subéchancrée ; bord droit un peu épais, sillonné à l'intérieur et ailé à la manière des strombes : couleur rousse, rougeâtre, alternativement fasciée de rouge et de brun.

De l'Océan indien.

Il paraît que c'est une coquille fort rare.

\bigskip

La \textbf{P. bizonale} ; \emph{P. bizonalis}, de Lamk., 7, 249, 49. Coquille fort petite, très-épaisse, lisse, ovale, globuleuse, à spire courte, obtuse ; ouverture lisse, terminée par un canal fort court : couleur jaune, avec deux bandes transverses blanches.

Patrie inconnue.

\bigskip

La \textbf{Pourpre noyau} : \emph{P. nucleus}, \emph{Bucc. nucleus}, Brug., Dict. enc., n.° 14. Coquille petite, lisse, luisante, striée en travers auprès et en dedans du bord droit ; ouverture arrondie : couleur d'un brun châtain.

Patrie inconnue.

\bigskip

D'après les caractères que nous venons de donner des différentes espèces de coquilles que M. de Lamarck a placées dans ce genre, il est évident que c'est un genre assez artificiel et mal circonscrit ; aussi M. de Lamarck lui-même a-t-il regardé comme des pourpres des espèces que d'abord il avait placées dans des genres voisins. Quoi qu'il en soit, les espèces pourraient être partagées par petits groupes qui en faciliteraient la connaissance. Le premier comprendrait celles qui ont les caractères de la pourpre persique, avec la forme particulière de son opercule, comme la pourpre patulée ; et le dernier renfermerait les espèces dont la spire est assez élevée pour leur avoir mérité l'épithète de scalariforme, comme la pourpre cabestan et plusieurs autres. On placerait ensuite dans des sections intermédiaires les espèces ovales, cerclées, comme la pourpre interrompue ; les espèces épineuses, comme les pourpres bourgeonnée, marron d'Inde, etc., qui rappellent les ricinules ; et enfin, les espèces ovales, alongées, plus ou moins buccinoïdes, comme les pourpres squameuse, teinturière, etc. On trouverait même probablement convenable de former une section distincte avec les pourpres guirlande et Francolin, qui ont en effet des caractères assez particuliers. (De B.)
\end{document}
